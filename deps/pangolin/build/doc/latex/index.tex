Pangolin is a lightweight portable rapid development library for managing Open\+GL display / interaction and abstracting video input. At its heart is a simple Open\+Gl viewport manager which can help to modularise 3D visualisation without adding to its complexity, and offers an advanced but intuitive 3D navigation handler. Pangolin also provides a mechanism for manipulating program variables through config files and ui integration, and has a flexible real-\/time plotter for visualising graphical data.

The ethos of Pangolin is to reduce the boilerplate code that normally gets written to visualise and interact with (typically image and 3D based) systems, without compromising performance. It also enables write-\/once code for a number of platforms, currently including Windows, Linux, O\+SX, Android and I\+OS.

\subsection*{Code}

Find the latest version on \href{http://github.com/stevenlovegrove/Pangolin}{\tt Github}\+:


\begin{DoxyCode}
1 git clone https://github.com/stevenlovegrove/Pangolin.git
\end{DoxyCode}


Current build status on \href{https://drone.io/github.com/stevenlovegrove/Pangolin}{\tt Drone.\+io} 

\subsection*{Dependencies}

Optional dependencies are enabled when found, otherwise they are silently disabled. Check the C\+Make configure output for details.

\subsubsection*{Required Dependencies}


\begin{DoxyItemize}
\item Open\+GL (Desktop / ES / E\+S2)
\item Glew
\begin{DoxyItemize}
\item (win) built automatically
\item (deb) sudo apt-\/get install libglew-\/dev
\item (mac) sudo port install glew
\end{DoxyItemize}
\item C\+Make (for build environment)
\begin{DoxyItemize}
\item (win) \href{http://www.cmake.org/cmake/resources/software.html}{\tt http\+://www.\+cmake.\+org/cmake/resources/software.\+html}
\item (deb) sudo apt-\/get install cmake
\item (mac) sudo port install cmake
\end{DoxyItemize}
\end{DoxyItemize}

\subsubsection*{Recommended Dependencies}


\begin{DoxyItemize}
\item Boost (optional with C++11. Configure with \textquotesingle{}cmake -\/\+D\+C\+P\+P11\+\_\+\+N\+O\+\_\+\+B\+O\+O\+ST=1 ..\textquotesingle{} )
\begin{DoxyItemize}
\item (win) \href{http://www.boost.org/users/download/}{\tt http\+://www.\+boost.\+org/users/download/}
\item (deb) sudo apt-\/get install libboost-\/dev libboost-\/thread-\/dev libboost-\/filesystem-\/dev
\item (mac) sudo port install boost
\end{DoxyItemize}
\item Python2 / Python3, for drop-\/down interactive console
\begin{DoxyItemize}
\item (win) \href{http://www.python.org/downloads/windows}{\tt http\+://www.\+python.\+org/downloads/windows}
\item (deb) sudo apt-\/get install libpython2.\+7-\/dev
\item (mac) preinstalled with osx
\end{DoxyItemize}
\end{DoxyItemize}

\subsubsection*{Optional Dependencies for video input}


\begin{DoxyItemize}
\item F\+F\+M\+P\+EG (For video decoding and image rescaling)
\begin{DoxyItemize}
\item (deb) sudo apt-\/get install ffmpeg libavcodec-\/dev libavutil-\/dev libavformat-\/dev libswscale-\/dev
\end{DoxyItemize}
\item D\+C1394 (For firewire input)
\begin{DoxyItemize}
\item (deb) sudo apt-\/get install libdc1394-\/22-\/dev libraw1394-\/dev
\end{DoxyItemize}
\item libuvc (For cross-\/platform webcam video input via libusb)
\begin{DoxyItemize}
\item git\+://github.com/ktossell/libuvc.\+git
\end{DoxyItemize}
\item libjpeg, libpng, libtiff, libopenexr (For reading still-\/image sequences)
\begin{DoxyItemize}
\item (deb) sudo apt-\/get install libjpeg-\/dev libpng12-\/dev libtiff5-\/dev libopenexr-\/dev
\end{DoxyItemize}
\item Open\+NI / Open\+N\+I2 (For Kinect / Xtrion / Primesense capture)
\item Depth\+Sense S\+DK
\end{DoxyItemize}

\subsubsection*{Very Optional Dependencies}


\begin{DoxyItemize}
\item Eigen / TooN (These matrix types supported in the Pangolin A\+PI.)
\item C\+U\+DA Toolkit $>$= 3.\+2 (Some C\+U\+DA header-\/only interop utilities included)
\begin{DoxyItemize}
\item \href{http://developer.nvidia.com/cuda-downloads}{\tt http\+://developer.\+nvidia.\+com/cuda-\/downloads}
\end{DoxyItemize}
\item Doxygen for generating html / pdf documentation.
\end{DoxyItemize}

\subsection*{Building}

Pangolin uses the C\+Make portable pre-\/build tool. To checkout and build pangolin in the directory \textquotesingle{}build\textquotesingle{}, enabling C++11 support instead of using Boost, execute the following at a shell (or the equivelent using a G\+UI)\+:


\begin{DoxyCode}
1 git clone https://github.com/stevenlovegrove/Pangolin.git
2 cd Pangolin
3 mkdir build
4 cd build
5 cmake -DCPP11\_NO\_BOOST=1 ..
6 make -j
\end{DoxyCode}


If you would like to build the documentation and you have Doxygen installed, you can execute\+:


\begin{DoxyCode}
1 make doc
\end{DoxyCode}


\subsection*{Issues}

Please visit \href{https://github.com/stevenlovegrove/Pangolin/issues}{\tt Github Issues} to view and report problems with Pangolin. Issues and pull requests should be raised against the devel branch which contains the current development version.

Please note; most Pangolin dependencies are optional -\/ to disable a dependency which may be causing trouble on your machine, simply blank out it\textquotesingle{}s include and library directories with a cmake configuration tool (e.\+g. ccmake or cmake-\/gui). 